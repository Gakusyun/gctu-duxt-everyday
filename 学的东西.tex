\documentclass[UTF8]{ctexart}
\title{高途考研每日督学笔记}
\author{Gakusyun}
\date{\today}
\usepackage{times}
\usepackage{graphicx}
\usepackage{fancyhdr}
\usepackage{multicol}
\usepackage[left=3.18cm, right=3.18cm, top=2.54cm, bottom=2.54cm]{geometry}
\usepackage{hyperref}
\hypersetup{hidelinks,
	colorlinks=true,
	allcolors=black,
	pdfstartview=Fit,
	breaklinks=true}
\pagestyle{fancy}  
\fancyhf{} % 清除默认页眉和页脚设置  
\fancyfoot[C]{\thepage} % 在页脚中央显示页码
\begin{document}
\maketitle
\thispagestyle{empty}
\newpage
\tableofcontents
\thispagestyle{empty}
\newpage
\setcounter{page}{1}
\section{7月24日}
\subsection{英语}
\subsubsection{长难句}
In particular, they called for forging closer collaborations with western state
governments, which are often uneasy with federal action, and with the private
landowners who control an estimated 95\% of the prairie chicken's habitat.

【难点分析】本句是复合句。句子主干为 they called for forging closer
collaborations with… and with…, 两个由 which 和 who 引导的定语从句分别修饰western state governments 和 the private landowners。

【主句翻译】他们呼吁与西部各州政府以及私人土地所有者缔造更紧密的合作关系。

【整句翻译】特别是,他们呼吁与西部各州政府以及私人土地所有者缔造更紧密的合作关系;西部各州政府通常不满联邦政府的行为,那些私人土地所有者控制着 95\%的草原榛鸡栖息地。
\subsubsection{词汇}
\begin{multicols}{2}
    call for 呼吁

    forge v. 形成

    collaboration with sb. 与……的合作

    uneasy adj. 忧虑的;不和谐的

    federal adj. 联邦政府的

    landowner n. 土地拥有者

    estimate v. 估计

    prairie chicken 北美草原松鸡

    habitat n.(动植物的)生活环境
\end{multicols}
\section{7月25日}
\subsection{英语}
\subsubsection{长难句}
A young man can decide on a likely spouse on his own and then ask his parents to arrange the marriage negotiations, or the young man's parents may make the choice of a spouse, giving the child little to say in the selection.
\subsubsection{词汇}
spouse n. 配偶

on one's own 独立地;独自

negotiation n. 谈判;协商

selection n. 选择
\subsection{数学}
\subsubsection{公式默写}
\begin{multicols}{3}
    $\sin\left ( \frac{\pi}{2} + \alpha  \right ) =\cos \alpha$

    $\cos\left ( \frac{\pi}{2}+\alpha\right )=-\sin \alpha$

    $\sin \left ( \pi - \alpha\right )=\sin \alpha$

    $\cos \left ( \pi - \alpha\right )=-\cos \alpha$

    $\sin^2\alpha=\frac{1-\cos 2\alpha}{2}$

    $\cos^2\alpha=\frac{1+\cos 2\alpha}{2}$
\end{multicols}
\end{document}