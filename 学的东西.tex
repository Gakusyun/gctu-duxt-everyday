\documentclass[UTF8]{ctexart}
\title{高途考研每日督学笔记}
\author{Gakusyun}
\date{\today}
\usepackage{times}
\usepackage{graphicx}
\usepackage{fancyhdr}
\usepackage{multicol}
\usepackage[left=3.18cm, right=3.18cm, top=2.54cm, bottom=2.54cm]{geometry}
\usepackage{hyperref}
\hypersetup{hidelinks,
	colorlinks=true,
	allcolors=black,
	pdfstartview=Fit,
	breaklinks=true}
\pagestyle{fancy}  
\fancyhf{} % 清除默认页眉和页脚设置  
\fancyfoot[C]{\thepage} % 在页脚中央显示页码
\begin{document}
\maketitle
\thispagestyle{empty}
\newpage
\tableofcontents
\thispagestyle{empty}
\newpage
\setcounter{page}{1}
\section{2024年7月24日}
\subsection{英语}
\subsubsection{长难句}
In particular, they called for forging closer collaborations with western state
governments, which are often uneasy with federal action, and with the private
landowners who control an estimated 95\% of the prairie chicken's habitat.

【难点分析】本句是复合句。句子主干为 they called for forging closer
collaborations with… and with…, 两个由 which 和 who 引导的定语从句分别修饰western state governments 和 the private landowners。

【主句翻译】他们呼吁与西部各州政府以及私人土地所有者缔造更紧密的合作关系。

【整句翻译】特别是,他们呼吁与西部各州政府以及私人土地所有者缔造更紧密的合作关系;西部各州政府通常不满联邦政府的行为,那些私人土地所有者控制着 95\%的草原榛鸡栖息地。
\subsubsection{词汇}
\begin{multicols}{2}
    call for 呼吁

    forge v. 形成

    collaboration with sb. 与……的合作

    uneasy adj. 忧虑的;不和谐的

    federal adj. 联邦政府的

    landowner n. 土地拥有者

    estimate v. 估计

    prairie chicken 北美草原松鸡

    habitat n.(动植物的)生活环境
\end{multicols}
\section{2024年7月25日}
\subsection{英语}
\subsubsection{长难句}
A young man can decide on a likely spouse on his own and then ask his parents to arrange the marriage negotiations, or the young man's parents may make the choice of a spouse, giving the child little to say in the selection.

1.本句为并列句,两个分句由并列连词 or 连接。

2.分句1中包含 and 连接的两个并列谓宾短语,其中谓语中的 decide on 意为“决定”。

3.分句2的结构为主干+现在分词短语,分词短语 giving the child little to say inthe selection 作状语表伴随。

参考翻译:年轻男性可以自己选定一位合适的配偶,然后请父母安排结婚事务的商讨;或者由年轻男性的父母选择配偶,在选择过程中孩子几乎没有发言权。
\subsubsection{词汇}
likely adj.可能的;有希望的;适合的 adv.可能

be likely to do sth. 可能做某事,倾向做某事

spouse n. 配偶

on one's own 独立地;独自

negotiation n. 谈判;协商

selection n. 选择
\subsection{数学}
\subsubsection{公式默写}
\begin{multicols}{3}
    $\sin\left ( \frac{\pi}{2} + \alpha  \right ) =\cos \alpha$

    $\cos\left ( \frac{\pi}{2}+\alpha\right )=-\sin \alpha$

    $\sin \left ( \pi - \alpha\right )=\sin \alpha$

    $\cos \left ( \pi - \alpha\right )=-\cos \alpha$

    $\sin^2\alpha=\frac{1-\cos 2\alpha}{2}$

    $\cos^2\alpha=\frac{1+\cos 2\alpha}{2}$
\end{multicols}
\section{2024年7月26日}
\subsection{英语}
\subsubsection{长难句}
They should start by discarding California's lame argument that exploring the contents of a smart-phone---a vast storehouse of digital information---is similar to, say, going through a suspect's purse.

1.本句为复合句。

2.句子主干为 They should start …argument + that 引导的同位语从句,其中 by discarding California’slame argument 为介词短语作方式状语,lame 意为“没有说
服力的、蹩脚的”。

3.同位语从句修饰 argument,用来说明其内容,从句谓语 is similar to 短语意为“与……相似”,两个破折号之间的内容为插入语,作同位语解释说明smartphone, say 作插入语,意为“比如说”。

【整句翻译】首先,他们应该摒弃加利福尼亚蹩脚的观点,即认为查看智能手机的内容——一个巨大的数字信息库——相当于,比如说,检查嫌疑犯的钱包。
\subsubsection{词汇}
discard v. 丢弃;抛弃;摒弃

storehouse n. 仓库;宝库

suspect n. 嫌疑犯

purse n. 钱包
\subsection{数学}
\subsubsection{概念}
用极限定义表示间断点的分类:
\begin{enumerate}
    \item 第一类间断点:$\lim\limits_{x \to x^{-}_{0}}f(x)$与$\lim\limits_{x \to x^{+}_{0}}f(x)$都存在的间断点.

          若$\lim\limits_{x \to x^{-}_{0}}f(x)=\lim\limits_{x \to x^{+}_{0}}f(x)$但不等于$f(x_0)$,或$f(x)$在$x=x_0$处无定义,则称$x_0$为可去间断点;

          若$\lim\limits_{x \to x^{-}_{0}}f(x)\ne\lim\limits_{x \to x^{+}_{0}}f(x)$,则称$x_0$为跳跃间断点。
    \item 第二类间断点:$\lim\limits_{x \to x^{-}_{0}}f(x)$与$\lim\limits_{x \to x^{+}_{0}}f(x)$中至少有一个不存在的间断点.

          若$\lim\limits_{x \to x^{-}_{0}}f(x)$与$\lim\limits_{x \to x^{+}_{0}}f(x)$中至少有一个为无穷大,则称$x_0$为无穷间断点;

          当$\lim\limits_{x \to x_{0}}f(x)$在某个范围内振荡,则称$x_0$为振荡间断点.

\end{enumerate}
\subsection{规划}
\subsubsection{政治}
25考研政治基础概念预热课1.1
\subsubsection{英语}
考研英语21天晨读1
\subsubsection{数学}
考研数学零基础提前学1并完成练习
\section{2024年7月27日}
\subsection{英语}
\subsection{数学}
\subsubsection{概念}
\begin{enumerate}
    \item 零点定理

          设函数$f(x)$在区间$[a,b]$上连续,且$f(a)$与$f(b)$异号,那么在开区间$(a,b)$内至少有一点$\xi$,

          使$f(\xi)=0$.
    \item 平均值定理

          设函数$f(x)$在区间$[a,b]$上连续,当$a<x_1<x_2<...<x_n<b$时,则在$[x_1,x_n]$内至少存在一点$\xi$,使$f(\xi)=\frac{f(x_1)+f(x_2)+...+f(x_n)}{n}$.
\end{enumerate}
\subsection{规划}
\end{document}